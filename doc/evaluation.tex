\documentclass{report}

\title{From Here to Beer: Demos and Evaluation}
\author{Ross Fenning}

\begin{document}

\maketitle

\chapter{Introduction}

Given the iterative, \emph{Agile} approach taken in developing the app,
user demoing and evaluation was able to take place at each iteration of the
software. It is with this regular feedback that the development is able
to adapt to user needs and suggestions. It is not uncommon for planned
features then to be dropped or prioritised differently after each
iteration demo.

It is with this methodology in mind that this user evaluation document
will walk through a series of small demos after implementing each feature.
This will show feedback on very early ``proof of concept'' stages of
the app through to a more finished product as well as how the development
direction was shaped by this feedback cycle.

In chapter~\ref{chapter:features} we will see this feedback cycle for
all features developed in the time allowed for building the app and
in chapter~\ref{chapter:evaluation} we will describe a final evalation
of the product attained at the end.

\chapter{Features}
\label{chapter:features}
\section{Random pub nearby}
\subsection{Requirements}
The first requirement was to develop the core proposition that a user may
request the name of a random pub nearby. Additional information other than
the pub name was out of scope as was any specific design. The intent was
to capture the main functionality in the first user story such that
the app -- whilst lacking aesthetic polish and other desirable features --
is usable for its main purpose.

\subsection{Demo}

The user recognised the first feature illustrated the proof of concept
well. The app was technically already usable so long as the named pub
was known to the user, but it was noted that it would be very important
to consider adding further information about the pub as it is capable
of naming one the user does not know.

The lack of any attention to layout or aesthetics was noted, but it was
accepted this first iteration was not focused on the design yet and it
still demonstrates the concept well. It was decided that the most
value would be gained from the next iteration by adding enough
additional pub information to help someone find a pub they do not know.

\section{ Address of the pub is given}
\subsection{Requirements}

The second requirement was simply to add additional information such as
the street address of the pub underneath its name when it is suggested.
Some thoughts around layout were to be considered as well.

\subsection{Demo}

Simply placing the pub name in the centre of the screen in a larger font
was seen to be an improvement over small text aligned in the top-left
by default. The address being in a smaller font was better for giving
visual separation of the pub name from the address.

The user saw the addition of the street address a very valuable improvement
as the app is somewhat more practical when you are told where the pub is.

\section{Show directions to the suggested pub}
\subsection{Requirements}

The idea behind the third feature was to expand on the usability gained
from the second feature by linking out to Google Maps to give walking
directions to the pub. Whilst the addition of the pub's address will
help someone look the pub up for themselves, it is a small improvement
and a convenience to the user to link to full directions with the press
of a single button.

\subsection{Demo}

The user was happy with the convenience gained by linking to Google Maps
with the press of a single button. This was deemed to make the app into
a usable state in that a pub can be suggested and directions to it given,
completing the user journey for the app's primary purpose. It was decided
that the next most useful improvement would be to allow some control over
the pub search radius.

\section{Ability to change radius of the pub search}
\label{sec:change-radius}
\subsection{Requirements}

This feature's aim was to allow the beginning of some level of customisation
of the app's main function. Some users may only want pubs suggested
immediately nearby and some may be keen on a longer stroll. This was to be
achieved with a circle on the first screen that the user can pinch-zoom
to reduce or expand the search radius.

\subsection{Demo}

The radius selector circle was deemed to be a useful feature and the fact
it changes colour from green (very near pubs) to red (pubs some distance
away) was noted as a positive addition.

It was also noted that it would be more useful if the circle were backed
by a static map to aid the visualisation of distance. This feature was
added to the backlog of potential improvements.

A bug was discovered where the app would crash if the radius were set
small enough such that no pubs are within that distance. The random
pub choice was not coping with the case where the list of pubs
from which to choose was empty.

\section{Ability to request another pub suggestion}
\label{sec:another-pub}
\subsection{Requirements}

The objective with this feature was to allow a user to ``roll again'' and
request another pub if they do not like a particular suggestion. It was
decided to implement this both as a button and also as an accelerometer
feature that detects the device being shaken.

\subsection{Demo}

This feature was welcomed as a useful addition. The ability to change
the suggestion to another one was seen as a core feature in order
for the app to be usable. The experience was noted to be quick and
easy to go through several suggestions.

One issue raised was that as the app progresses through several new
activities for each new suggestion, the only way to return to the main
screen with the radius selector is then by pressing the back button
several times.

Another issue is that there is no feedback as to how many pubs there are
in total in the search area, so it is not easy to tell if you are
getting the same suggestion multiple times due to randomness or
because there are not that many pubs within the radius given.

Both of these issues were added to the backlog.

\section{Saving favourite pubs}
\subsection{Requirements}

The purpose of this feature is to allow a basic ability to mark certain
pubs as favourites. When a favourite pub is then suggested at a later
date, the user is reminded that this is one they enjoyed. A list
was also added to allow deletion.

\subsection{Demo}

The user enjoyed the star icon marking a favourite pub and noted
that the feature worked as designed. The feature served its purpose
as a reminder well, but had no further purpose above that. The user
felt they would get more out of the feature if they were able to see
favourites marked by others and things like lists of most popular
pubs in an area. This would require an Internet-based service, but
it is a feature to consider for the longer term.

It was also noted that the favourite status of a pub did not factor
into the random suggestions. It was suggested that favourites could
appear more or less frequently and perhaps the user could have some
controls to adjust this.

\section{Crash when there are no pubs in range}
\subsection{Bug Description}

This was a bug raised in the demo in section~\ref{sec:change-radius}.
The random pub selection failed in the case where the list of
potential pubs was empty and the error was not handled well enough
to stop the app from crashing.

What was added was a dialogue prompt that would pop up to inform
the user if there are no pubs within the chosen radius. They would
then be invited to try a larger radius.

\subsection{Demo}

The user acknowledged the problem had been fixed. The only comment
was that the dialogue message could be friendlier or even make
more of a joke about the user not wanting to walk very far.

\section{Fixing the back button behaviour}

\subsection{Requirements}
In section~\ref{sec:another-pub}, it was noted that the back button
would take the user back through the whole trail of pub suggestions.
This meant there was no quick way to get back to the main
screen with the radius selector. This was a result of doing each
of the screens as a separate activity. The Android system was
automatically transitioning through a stack of new activities
and providing this trail.

It was decided it would be better to refactor the app to use
a single main activity with multiple fragments. The activity
itself can then have more control over transitions and the views
as well as the back button behavour. The back button was to
take the user back to the home screen if they are on a pub
suggestion and quit the app if they are on the main screen.

\subsection{Demo}

The user noted that the change removed a lot of confusion about
how to use the app.

\section{Add count of total number of pubs in search}

\subsection{Requirements}

This was raised in the demo in section~\ref{sec:another-pub}. When
the user was getting the same handful of suggestions within a small
search radius, it would have been a little more useful to indicate
how many pubs in total were involved in the search to give an
idea of how likely it is for the same suggestion to appear again.

A small amount of text was added underneath the pub's name and
address to indicate that it was chosen out of $N$ different
possible pubs in the search area.

\subsection{Demo}

The user expressed that it was a useful piece of information.

\section{Layout Changes}
\subsection{Requirements}
As the time allocated to developing the app was coming to an end, the
final feature here was to do some quick layout changes to make the
app a little more user-friendly. All of this was done with user
input and suggestions.

Firstly, the main ``From Here to Beer!'' button on the main screen
was made larger and moved to the bottom of the screen. This is
closer to where the user's thumb would be and makes it clearer
that this is the primary button to engage the app.

Secondly, some of the spacing between the three action buttons under
the pub (``Get Directions'', ``Save to Favourites'' and ``Take Me
Somewhere Else!'') was not consistent. This was tidied up.

Thirdly, the text ``Finding\dots'' that appeared while a pub
was being chosen was still in a small font aligned to the top-left.
This was made larger and centred.

\subsection{Demo}

The user deemed all of these to be small improvements to the aesthetics.
The larger ``Finding\dots'' text was especially noted to make
it clearer that a pub is being searched and the user should wait.

\chapter{Final Evaluation}
\label{chapter:evaluation}

\end{document}
