\documentclass{report}

\title{From Here to Beer: Demos and Evaluation}
\author{Ross Fenning}

\begin{document}

\maketitle

\chapter{Introduction}

Given the iterative, \emph{Agile} approach taken in developing the app,
user demoing and evaluation was able to take place at each iteration of the
software. It is with this regular feedback that the development is able
to adapt to user needs and suggestions. It is not uncommon for planned
features then to be dropped or prioritised differently after each
iteration demo.

It is with this methodology in mind that this user evaluation document
will walk through a series of small demos after implementing each feature.
This will show feedback on very early ``proof of concept'' stages of
the app through to a more finished product as well as how the development
direction was shaped by this feedback cycle.

In chapter~\ref{chapter:features} we will see this feedback cycle for
all features developed in the time allowed for building the app and
in chapter~\ref{chapter:evaluation} we will describe a final evalation
of the product attained at the end.

\chapter{Features}
\label{chapter:features}
\section{Feature \#001: Random pub nearby}
\subsection{Requirements}
The first requirement was to develop the core proposition that a user may
request the name of a random pub nearby. Additional information other than
the pub name was out of scope as was any specific design. The intent was
to capture the main functionality in the first user story such that
the app -- whilst lacking aesthetic polish and other desirable features --
is usable for its main purpose.

\subsection{Demo}

The user recognised the first feature illustrated the proof of concept
well. The app was technically already usable so long as the named pub
was known to the user, but it was noted that it would be very important
to consider adding further information about the pub as it is capable
of naming one the user does not know.

The lack of any attention to layout or aesthetics was noted, but it was
accepted this first iteration was not focused on the design yet and it
still demonstrates the concept well. It was decided that the most
value would be gained from the next iteration by adding enough
additional pub information to help someone find a pub they do not know.

\section{Feature \#002: Address of the pub is given}
\subsection{Requirements}

The second requirement was simply to add additional information such as
the street address of the pub underneath its name when it is suggested.
Some thoughts around layout were to be considered as well.

\subsection{Demo}

Simply placing the pub name in the centre of the screen in a larger font
was seen to be an improvement over small text aligned in the top-left
by default. The address being in a smaller font was better for giving
visual separation of the pub name from the address.

The user saw the addition of the street address a very valuable improvement
as the app is somewhat more practical when you are told where the pub is.

\section{Feature \#003: Show directions to the suggested pub}
\subsection{Requirements}

The idea behind the third feature was to expand on the usability gained
from the second feature by linking out to Google Maps to give walking
directions to the pub. Whilst the addition of the pub's address will
help someone look the pub up for themselves, it is a small improvement
and a convenience to the user to link to full directions with the press
of a single button.

\subsection{Demo}

The user was happy with the convenience gained by linking to Google Maps
with the press of a single button. This was deemed to make the app into
a usable state in that a pub can be suggested and directions to it given,
completing the user journey for the app's primary purpose. It was decided
that the next most useful improvement would be to allow some control over
the pub search radius.

\section{Feature \#004: Ability to change radius of the pub search}
\subsection{Requirements}

This feature's aim was to allow the beginning of some level of customisation
of the app's main function. Some users may only want pubs suggested
immediately nearby and some may be keen on a longer stroll. This was to be
achieved with a circle on the first screen that the user can pinch-zoom
to reduce or expand the search radius.

\subsection{Demo}

\section{Feature \#005: Ability to request another pub suggestion}
\subsection{Requirements}
\subsection{Demo}

\section{Feature \#006: Saving favourite pubs}
\subsection{Requirements}
\subsection{Demo}

\chapter{Final Evaluation}
\label{chapter:evaluation}

\end{document}
